\documentclass[10pt,twocolumn,letterpaper]{article}

\usepackage{cvpr}
\usepackage{times}
\usepackage{epsfig}
\usepackage{graphicx}
\usepackage{amsmath}
\usepackage{amssymb}

% Include other packages here, before hyperref.

% If you comment hyperref and then uncomment it, you should delete
% egpaper.aux before re-running latex.  (Or just hit 'q' on the first latex
% run, let it finish, and you should be clear).
\usepackage[breaklinks=true,bookmarks=false]{hyperref}

\DeclareGraphicsExtensions{.pdf,.jpg,.png,.jpeg}
\graphicspath{{images/}, {figs/}}
\newcommand{\todo}[1]{\textcolor{red}{{\em [#1]}} }
\newcommand{\specialcell}[2][c]{%
    \begin{tabular}[#1]{@{}c@{}}#2\end{tabular}}
\newcommand{\secref}[1]{Section~\ref{sec:#1}}
\newcommand{\figref}[1]{Figure~\ref{fig:#1}}
\newcommand{\tabref}[1]{Table~\ref{tab:#1}}
\newcommand{\equref}[1]{Equation~\ref{equ:#1}}
\newcommand{\matr}[1]{\mathbf{#1}}
\newcommand{\bfit}[1]{\mathbf{\mathit{#1}}}
\newcommand{\bs}[1]{\boldsymbol{#1}}

\DeclareMathOperator*{\argmax}{arg\,max}
\DeclareMathOperator*{\argmin}{arg\,min}

\cvprfinalcopy % *** Uncomment this line for the final submission

\def\cvprPaperID{****} % *** Enter the CVPR Paper ID here
\def\httilde{\mbox{\tt\raisebox{-.5ex}{\symbol{126}}}}

% Pages are numbered in submission mode, and unnumbered in camera-ready
%\ifcvprfinal\pagestyle{empty}\fi
\begin{document}

%%%%%%%%% TITLE
\title{Survey On Image and Video Captioning}

\author{Weilian Song\\
University of Kentucky\\
{\tt\small weilian.song@uky.edu}
% For a paper whose authors are all at the same institution,
% omit the following lines up until the closing ``}''.
% Additional authors and addresses can be added with ``\and'',
% just like the second author.
% To save space, use either the email address or home page, not both
}

\maketitle
%\thispagestyle{empty}

%%%%%%%%% BODY TEXT
\section{Introduction}

Automatic image and video captioning is a relatively new field in the field of
Machine Learning, where a program can automatically assign a
sentenced-description of an image, or in the case of a video, for every frame
of the video. Desite being new, this technology has improved itself over the
past three years and is widely used in today's world, from auto-generating
video captions for Youtube videos to clustering images/videos based on their
captions.

In this survey, we first explore the basic concepts of a nerual image/video
caption generator, which include topics like Recurrent Neural Networks (RNN),
Long-Short-Term-Memory (LSTM), and a naive model for performing such task.
\todo{cite?} We will also explain various evaluation methods that authors use
to compare their models with others. \todo{figure out their eval methods}

Next we explore two approaches for improving the accuracy of the naive method,
one through the use of an attention mechanism, and the other through the
process of dense labelling. Both of these methods are imporved upon in this
past year's CVPR conference, with new state-of-the-art results within their
respective field.

Finally, we will be looking at two very interesting papers that offer a fresh
look on captioning and piqued my interest in the very beginning.

%-------------------------------------------------------------------------
\section{Background Information}

Before we deep dive into the six papers, some clarifications need to be made
on various topics.

\subsection{Multilayer Perceptrons}
\todo{Cite the data science blog post on MLP}

Multilayer perceptron, or MLP for short, is the building block to a lot of
modern neural networks. MLP consists of many layers of neurons, with each
neuron connecting to every input node and every output node, associating each
edge with a weight, as shown in \todo{add a MLP figure here}. In a simple,
non-recurrent MLP, the connected nodes together make an acyclic graph with the
left side as inputs and right side as outputs. 

When we provide an input, each neuron computes a weighted sum of all of its
inputs based on its weights, and combined with a bias term, the output passes
through an activation function, which introduces non-linearity, and becomes
the neuron's output. Connected neurons are then updated as well, until the
final layer is reached and the output of the neurons will be the network
outputs. \todo{maybe more information?}

\subsection{Recurrent Neural Networks}
\todo{Cite deeplearning4j recurrent}

Recurrent Neural Networks, or RNN for short, are a class of neural networks
that make use of previous network outputs for the next immediate prediction.
They are able to maintain long-term dependencies between predictions, which is
useful in fields like nautral language processing, where prediction at the
current time step requires context, which is obtained through previous
predictions. 

To define it more formally, assuming we have a recurrent neural network of
only one layer with input $\matr{x}$, layer weights $W$, and output
$\bs{y}_{t-1}$ at time step $t-1$, we can define our network output
$\bs{y}$ at time step $t$ as:
%
\begin{equation}
  \bs{y}_t = \phi (W \matr{x_t} + U \bs{y}_{t-1})
\end{equation}
%
where $\phi$ is an activation function that squashes the values in the
parentheses into a specified range, and $U$ is the transition matrix between
hidden states and the concept of hidden-states to hidden-states is similar to
a Markov chain.

As we can see, our network output at timestep $t$ is depend upon the regular
feedforward section of the network $W \matr{x_{t}}$ and some information about
its past outputs, which is obtained with $U \bs{y}_{t-1}$. $U$ is a
learnable matrix that learns what and how much previous outputs should the
network apply to the immediate future prediction.

Extending this simple example to a network with multiple and different types
of layers, our layer's output will not be $\bs{y}$ anymore, instead it
will become hidden, not directly observable, which we denote as
$\bs{h}$. The rest of the equation remains the same, and multiple
recurrent neural layers can be stacked to learn more complex 

When backpropagating the error through the network and updating the learnable
parameters of the recurrent layers, the gradient calculation is comparable to
a feed-forward network, except the fact that the layer's output is the result
of a series of chain rules. Calculus is again used but no special exceptions
are made for recurrent layers.

One major issue with vanilla RNNs is that during training, the weights of a
recurrent layer can change too rapidly from optimization, and the gradient,
the rate of change of weights, can also change too rapidly, causing each step
afterwards to amplify the change, and evantually either explodes or vanishes.
This is known as the vanishing or exploding gradient problem, and because of
it LSTM units are born.

\subsection{Long Short-Term Memory Units}
\todo{Cite colah's guide on LSTM}

Long Short-Term Memory units, or LSTM for short, is a special kind of
recurrent neural layer. Instead of only using $U$ to determine what
information to keep between states, it uses a series of gates to learn what to
remember from the previous state, use the information obtained, and decides
what to forget so the next state can have the most useful information.
\todo{explain more about LSTMs here}

\subsection{Evaluation Methods}
For a given image, there usually are several reference captions that the
network can try to predict. For evaluation of image/video captioning systems,
we will be looking mainly at three today: BLEU, METEOR and CIDEr.

\paragraph{BLEU}
BLEU aims at computing the similarity between the candidate (predicted)
sentences and all reference sentences by finding the maximum number of
appearances of $n$-grams of one candidate sentence inside the corresponding
reference sentences.  \todo{sounds weird} An $n$-gram is any given phrase that
is $n$ words long. As an example, consider the candidate and reference text in
\tabref{captions}:
%
\begin{table}[]
\centering
\caption{Example candidate and reference captions}
\label{tab:captions}
\begin{tabular}{ll}
	Candidate   & the the the the the the the \\
	Reference 1 & the cat is on the mat       \\
  Reference 2 & there is a cat on the mat  
\end{tabular}
\end{table}
%
If we are to compute the 1-gram BLEU score (BLEU-1), it would be $2/7$, as the
only 1-gram ``the'' in candidate appears a maximum of two times in Reference
2. 

\paragraph{METEOR}


\paragraph{CIDEr}

\section{Attention Mechanism}

The use of an attention mechanism has become popular in image and video
captioning, as it assists the neural network to extract more useful
information by focusing its attention to only parts of the input. Below are
two different attention mechanisms introduced in CVPR 2017, and we will be
taking a look at each one and compare their results on the COCO dataset.
\todo{cite COCO here}

\subsection{Knowing When To Look}
For a typical attention mechanism, it knows what to look, but is forced to
look all the time. Lu et.\ al.\ tackle the question of not just \textit{where}
to look but \textit{when} to look as well. In their proposed model, an encoder
takes in an input image and outputs a context vector $\bs{c}_t$, which
is the output of a trained image classification network and provides visual
features for caption generating. $\bs{c}_t$ is then fed into a LSTM
unit, followed by a novel attention mechanism and a MLP for the final
prediction. \todo{maybe a diagram here? Also this is wrong}

The novel attention mechanism that Lu et. al. introduces is a variation of
soft attention model from ??? \todo{cite}. To compute the context vector
$\bs{c}_t$, they define it as:
%
\begin{equation}
  \bs{c}_t = g(\matr{V}, \bs{h}_t)
\end{equation}
%
where $\matr{V} = [\mathbf{v}_1,\dots,\mathbf{v}_k], \mathbf{v}_i \in R^d$
\todo{fix real numbers} is the image feature divided into $k$ areas and each
area having $d$ dimensions. $g$ is the attention mechanism the author
proposes, which is a single layer neural network computing the attention
distrubtion of the $k$ areas of the image:
%
\begin{align}
  \matr{z}_t &= \matr{w}_h^T \tanh(\matr{W}_v \matr{V} + \matr{W}_g \matr{h}_t) \\
  \alpha_t   &= \operatorname{softmax}(\matr{z}_t)
\end{align}
%
where \todo{a lot of explaining here...}, and $\alpha \in R^k$ is the
attention weight over $k$ areas of the image feature $\matr{V}$. With
$\alpha_t$, we can compute the context vector as:
%
\begin{equation}
  \bs{c}_t = \sum_{i=1}^k \alpha_{ti} \bs{v}_{ti}
\end{equation}
%
and finally we have a MLP layer $f$ that computes the probability distribution
over words:
%
\begin{equation}
  \log p(y_t | y_1,\dots,y_{t-1},I) = f(\bs{h}_t, \bs{c}_t)
\end{equation}
%
Overview of the attention network is shown in ???, which is borrowed from the
original text.

Authors also propose a method for allowing the network to choose features
produced by the attention system or another set of features
$\bs{s}_t$, the ``visual sentinel''. It is calculated as follows:
%
\begin{align}
  \bs{g}_t &= \sigma (\bfit{W}_x \bs{x}_t + \bfit{W}_h \bs{h}_{t-1} \\
  \bs{s}_t &= \bs{g}_t \odot \tanh (\bs{m}_t)
\end{align}
%
where $\bfit{W}_x$ and $\bfit{W}_h$ are again weights to be learned and
$\bs{m}_t$ is the LSTM cell state. Combining with the attention
mechanism output $\bs{c}_t$, we have:
%
\begin{equation}
  \hat{\bs{c}}_t = \beta_t \bs{s}_t + (1 - \beta_t) \bs{c}_t
\end{equation}
%
where $\beta_t$ is our sentinel gate and is between 0 and 1 to control how
much information to use from each. When $\beta_t=1$, we essentially turn
off the attention mechanism and only use the visual sentinel feature, vice
versa. \todo{maybe explain that it's inspired by residual nets}

\subsection{Areas of Attention for Image Captioning}

\begin{itemize}
  \item Uses regions of image to predict the next word, and uses predicted
        regions in the next state of the RNN cell.
  \item Have three different types of regions of interest. Either grid, object
        bounding box, or spatial transformers.
  \item Uses VGG16
\end{itemize}

\section{Dense Labeling}

Dense labeling

\section{Miscellaneous}

Miscellaneous papers I want to talk about

\section{Conclusion}

Conclusion

{\small
\bibliographystyle{ieee}
\bibliography{egbib}
}

\end{document}
