\documentclass[10pt,twocolumn,letterpaper]{article}

\usepackage{cvpr}
\usepackage{times}
\usepackage{epsfig}
\usepackage{graphicx}
\usepackage{amsmath}
\usepackage{amssymb}

% Include other packages here, before hyperref.

% If you comment hyperref and then uncomment it, you should delete
% egpaper.aux before re-running latex.  (Or just hit 'q' on the first latex
% run, let it finish, and you should be clear).
\usepackage[breaklinks=true,bookmarks=false]{hyperref}

\cvprfinalcopy % *** Uncomment this line for the final submission

\def\cvprPaperID{****} % *** Enter the CVPR Paper ID here
\def\httilde{\mbox{\tt\raisebox{-.5ex}{\symbol{126}}}}

% Pages are numbered in submission mode, and unnumbered in camera-ready
%\ifcvprfinal\pagestyle{empty}\fi
\setcounter{page}{4321}
\begin{document}

%%%%%%%%% TITLE
\title{Survey On Image and Video Captioning}

\author{Weilian Song\\
University of Kentucky\\
{\tt\small weilian.song@uky.edu}
% For a paper whose authors are all at the same institution,
% omit the following lines up until the closing ``}''.
% Additional authors and addresses can be added with ``\and'',
% just like the second author.
% To save space, use either the email address or home page, not both
}

\maketitle
%\thispagestyle{empty}

%%%%%%%%% ABSTRACT
\begin{abstract}
  Lol probably not an abstract
\end{abstract}

%%%%%%%%% BODY TEXT
\section{Introduction}

Introduction

%-------------------------------------------------------------------------
\section{Background Information}

Background Information

Possibly some stuff on RNN in a small section?

\section{Attention Mechanism}

Attention Mechanism

\subsection{Knowing When To Look Notes}

\begin{itemize}
  \item The network learns when to pay attention to images and when to pay
        attention to learned words. Uses gate $g_t$, which is learnable
  \item Novel attention mechanism for visual learning, inspired by success in
        residual learning. Different from other methods, as LSTM's cell state
        is used for learning
  \item Extensive evaluation on the visuan sentinel and the attention mechanism
\end{itemize}

\subsection{Areas of Attention for Image Captioning}

\begin{itemize}
  \item Uses regions of image to predict the next word, and uses predicted
        regions in the next state of the RNN cell.
  \item Have three different types of regions of interest. Either grid, object
        bounding box, or spatial transformers.
  \item Uses VGG16
\end{itemize}

\section{Dense Labeling}

Dense labeling

\section{Miscellaneous}

Miscellaneous papers I want to talk about

\section{Conclusion}

Conclusion

{\small
\bibliographystyle{ieee}
\bibliography{egbib}
}

\end{document}
